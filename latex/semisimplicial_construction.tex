\documentclass[notitlepage]{scrartcl}

\usepackage[parfill]{parskip}
\usepackage{unicode-math}
\usepackage{fontspec}
\setmonofont[Scale=0.8]{DejaVu Sans Mono}

\newcommand{\code}{\texttt}

\newcommand{\SST}{\mathrm{SST}}
\newcommand{\Sk}{\mathrm{Sk}}

\title{The semisimplicial type construction}
\author{}
\date{\vspace{-5ex}\today}

\begin{document}

\maketitle
\begin{abstract}
A careful writeup of the gory details of the semisimplicial type construction in this internal CwF development.
\end{abstract}

We mutually define

\begin{verbatim}
SST₋ : ℕ → Con
X    : (m n : ℕ) → {m ≤ n} → Ty (SST₋ (S n))
A    : (m n : ℕ) {h : m ≤ n} → Tm (X m n {h})
Sk   : (k n : ℕ) → {{O < n}} → k < n → Ty (SST₋ n)
\end{verbatim}

and \code{SST : ℕ → Con} where \code{SST n = SST₋ (S n)}.

Intuitively, these are supposed to be the following:
$$\SST_n = A_0 \colon X_0, \dotsc, A_n \colon X_n$$
is the context of $k$-simplex fillers $A_k$ for $0 \leq k \leq n$.
$X_0 = U$ and $X_k = \Sk(k-1, k) \rightarrow U$ for $0 < k \leq n$, where $\Sk(k-1, k)$ is the $(k-1)$-skeleton of $\Delta^k$.

\code{SST n} is meant to be the context
\begin{verbatim}
A O n : X O n ∷
A 1 n : X 1 n ∷
     ...      ∷
A n n : X n n
\end{verbatim}


\end{document}